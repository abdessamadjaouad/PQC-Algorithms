% ============================================================================
% CHAPTER 1: IoT AND SECURITY CHALLENGES
% ============================================================================

\chapter{IoT and Security Challenges}
\label{chap:iot_security}

\section{Introduction}

The Internet of Things (IoT) connects billions of devices worldwide, from sensors to industrial machines. These devices face severe resource constraints and an emerging quantum computing threat that will break current encryption. This chapter examines both challenges.

\begin{figure}[H]
\centering
\includegraphics[width=0.95\textwidth]{figures/iot_architecture.png}
\caption{Three-layer IoT architecture showing the perception, network, and application layers}
\label{fig:iot_architecture}
\end{figure}

\section{IoT Device Constraints}

IoT devices are designed to be small, cheap, and energy-efficient, creating significant constraints for security implementation.

\subsection{Energy Constraints}

Many IoT devices run on batteries for months or years. Every operation consumes energy: computation, communication, and memory access. Communication is particularly expensive, making data compression essential.

\begin{table}[H]
\centering
\caption{Energy consumption of common IoT operations}
\label{tab:energy_operations}
\begin{tabular}{lcc}
\toprule
\textbf{Operation} & \textbf{Energy (mJ)} & \textbf{Relative Energy Cost} \\
\midrule
Send 1 KB over WiFi & 10--100 & Medium \\
Send 1 KB over LoRa & 150--500 & High \\
AES-128 encryption (1 KB) & 0.001--0.01 & Low \\
RSA-2048 signature & 0.1--1.0 & Medium \\
\bottomrule
\end{tabular}
\end{table}


\subsection{Memory and Bandwidth Limitations}

Typical IoT microcontrollers have 16--256 KB RAM and 128 KB--1 MB flash. This constrains cryptographic implementation since keys and working space must fit in limited memory.

\begin{table}[H]
\centering
\caption{Memory requirements: Classical vs Post-Quantum Cryptography}
\label{tab:memory_comparison}
\begin{tabular}{lccc}
\toprule
\textbf{Algorithm} & \textbf{Public Key} & \textbf{Private Key} & \textbf{Ciphertext} \\
\midrule
ECC P-256 & 64 bytes & 32 bytes & 64 bytes \\
Kyber768 & 1,184 bytes & 2,400 bytes & 1,088 bytes \\
\bottomrule
\end{tabular}
\end{table}

Low-power protocols like LoRa offer only 0.3--50 Kbps, making every byte critical.

\begin{table}[H]
\centering
\caption{Bandwidth of common IoT protocols}
\label{tab:iot_protocols}
\begin{tabular}{lcc}
\toprule
\textbf{Protocol} & \textbf{Data Rate} & \textbf{Range} \\
\midrule
WiFi (802.11n) & 150 Mbps & 50 m \\
Bluetooth LE & 1 Mbps & 10 m \\
Zigbee & 250 Kbps & 100 m \\
LoRa & 0.3 - 50 Kbps & 15 km \\
\bottomrule
\end{tabular}
\end{table}

\begin{figure}[H]
\centering
\includegraphics[width=0.75\textwidth]{figures/iot_constraints.png}
\caption{Resource capabilities across IoT device classes}
\label{fig:iot_constraints}
\end{figure}

Any security solution must be lightweight, efficient, and fast.

\section{The Quantum Threat}

\subsection{Quantum Computing and Cryptography}

Quantum computers use qubits that can be in superposition, allowing them to explore many solutions simultaneously. Shor's algorithm \cite{shor1994algorithms} can efficiently solve integer factorization and discrete logarithm problems---the foundations of RSA, ECC, and Diffie-Hellman.

\begin{table}[H]
\centering
\caption{Impact of quantum computing on cryptographic algorithms}
\label{tab:quantum_impact}
\begin{tabular}{llc}
\toprule
\textbf{Algorithm} & \textbf{Type} & \textbf{Quantum Security} \\
\midrule
RSA / ECC / ECDSA & Asymmetric & \textcolor{red}{Broken} \\
AES-256 & Symmetric & \textcolor{green}{Secure} \\
SHA-384 & Hash function & \textcolor{green}{Secure} \\
\bottomrule
\end{tabular}
\end{table}

\subsection{Timeline and Urgency}

Cryptographically relevant quantum computers are expected within 10--20 years \cite{preskill2018quantum}. However, the ``harvest now, decrypt later'' threat means data encrypted today may be compromised once quantum computers arrive. IoT devices with 10--15 year operational lifetimes are already at risk.

% \begin{figure}[H]
% \centering
% \fbox{
% \begin{minipage}{0.9\textwidth}
% \centering
% \vspace{0.3cm}
% \textbf{Quantum Computing Timeline}\\[0.3cm]
% \begin{tabular}{cl}
% \textbf{2024} & NIST finalizes PQC standards (Kyber, Dilithium) \\
% \textbf{2025} & Transition to PQC should begin \\
% \textbf{2030-2040} & Cryptographically relevant quantum computers expected \\
% \end{tabular}
% \vspace{0.3cm}
% \end{minipage}
% }
% \caption{Timeline of quantum computing development}
% \label{fig:quantum_timeline_text}
% \end{figure}

\begin{figure}[H]
\centering
\includegraphics[width=1\textwidth]{figures/quamtum_history.png}
\caption{Timeline of quantum computing development}
\label{fig:quantum_timeline_text}
\end{figure}

\section{Chapter Conclusion}

IoT devices face severe resource constraints: limited energy, memory (kilobytes), and bandwidth (especially for LoRa). Simultaneously, quantum computers will break RSA and ECC. The combination creates our research problem: implementing quantum-resistant security on constrained devices while managing PQC's larger key and ciphertext sizes. Chapter 2 explores Post-Quantum Cryptography solutions.
