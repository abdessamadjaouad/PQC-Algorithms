% ============================================================================
% GENERAL INTRODUCTION
% ============================================================================

\chapter*{General Introduction}
\addcontentsline{toc}{chapter}{General Introduction}

\section*{Context and Motivation}

The Internet of Things (IoT) connects billions of devices worldwide, from sensors to industrial machines. However, these devices face severe resource constraints---limited energy, memory, and bandwidth---making strong security difficult to implement. 

A new threat compounds this challenge: quantum computers. Algorithms like Shor's can break current RSA and ECC encryption, potentially compromising all data encrypted today through ``harvest now, decrypt later'' attacks.

\section*{Problem Statement}

Post-Quantum Cryptography (PQC) offers quantum-resistant security. In 2024, NIST standardized Kyber for key exchange and Dilithium for signatures. However, PQC algorithms produce significantly larger keys---Kyber768's public key (1,184 bytes) is 4.6× larger than RSA-2048 (256 bytes). This size overhead conflicts with IoT bandwidth constraints.

\begin{center}
\textit{\textbf{How can we secure IoT communications against quantum threats while respecting bandwidth constraints?}}
\end{center}

\section*{Research Objectives and Contributions}

This research aims to:
\begin{enumerate}
    \item Analyze IoT constraints and evaluate PQC algorithms for suitability
    \item Investigate compression techniques to reduce PQC overhead
    \item Design a combined compress-then-encrypt architecture
    \item Implement and benchmark the approach
\end{enumerate}

\noindent Key contributions include:
\begin{itemize}
    \item Selection of Kyber768 as optimal PQC for IoT (security level 3, balanced parameters)
    \item Selection of LZ4 compression for minimal memory footprint (16 KB) and energy efficiency
    \item A working implementation achieving significant bandwidth savings
\end{itemize}

\section*{Thesis Organization}

\begin{itemize} 
    \item \textbf{Chapter 1} examines IoT constraints and the quantum threat.
    \item \textbf{Chapter 2} presents PQC with focus on Kyber768.
    \item \textbf{Chapter 3} covers compression algorithms, emphasizing LZ4.
    \item \textbf{Chapter 4} presents our combined architecture.
    \item \textbf{Chapter 5} provides implementation and benchmarks.
    \item \textbf{General Conclusion} summarizes findings and future directions.
\end{itemize}

\newpage
