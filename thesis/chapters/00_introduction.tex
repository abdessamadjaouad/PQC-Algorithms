% ============================================================================
% GENERAL INTRODUCTION
% ============================================================================

\chapter*{General Introduction}
\addcontentsline{toc}{chapter}{General Introduction}

\section*{Context and Motivation}

The Internet of Things (IoT) connects billions of devices worldwide, from sensors to industrial machines. However, these devices face severe resource constraints---limited energy, memory, and bandwidth---making strong security difficult to implement. Simultaneously, quantum computers pose an emerging threat: algorithms like Shor's can break current RSA and ECC encryption, potentially compromising all data encrypted today through ``harvest now, decrypt later'' attacks.

\section*{Problem Statement}

Post-Quantum Cryptography (PQC) offers quantum-resistant security, with NIST standardizing Kyber for key exchange in 2024. However, PQC algorithms produce significantly larger keys and ciphertexts than classical cryptography---Kyber768's public key (1,184 bytes) is 18 times larger than ECC P-256 (64 bytes). This size overhead conflicts directly with IoT bandwidth constraints.

\begin{center}
\textit{\textbf{How can we secure IoT communications against quantum threats while respecting bandwidth constraints?}}
\end{center}

\section*{Research Objectives and Contributions}

This thesis makes the following contributions:
\begin{enumerate}
    \item Analysis of PQC algorithms for IoT, focusing on Kyber768 as the optimal choice
    \item Evaluation of compression algorithms, selecting LZ4 for its energy efficiency and minimal memory footprint on constrained devices
    \item A combined compress-then-encrypt architecture achieving significant bandwidth savings
    \item Working implementation with benchmarks demonstrating practical viability
\end{enumerate}

\section*{Thesis Organization}

\textbf{Chapter 1} examines IoT constraints and the quantum threat. \textbf{Chapter 2} presents Post-Quantum Cryptography, focusing on Kyber768. \textbf{Chapter 3} covers compression algorithms with emphasis on LZ4. \textbf{Chapter 4} presents our combined architecture. \textbf{Chapter 5} provides implementation and benchmarks. The \textbf{General Conclusion} summarizes findings and future directions.

\newpage
