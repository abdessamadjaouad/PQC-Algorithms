% ============================================================================
% GENERAL CONCLUSION
% ============================================================================

\chapter*{General Conclusion}
\addcontentsline{toc}{chapter}{General Conclusion}

This thesis addressed a critical challenge: securing resource-constrained IoT devices against quantum computing threats while maintaining the efficiency required for practical deployment.

\section*{Summary of Contributions}

\textbf{First}, we analyzed IoT security requirements in the quantum threat context, establishing criteria that quantum-resistant solutions must satisfy for devices with limited processing power, memory, bandwidth, and battery life.

\textbf{Second}, we evaluated compression algorithms for IoT, identifying LZ4 as optimal for resource-constrained endpoints due to its 16 KB memory footprint, extreme speed (500+ MB/s), and acceptable compression ratios---factors critical for battery-powered sensors where computational energy must be minimized.

\textbf{Third}, we developed a combined architecture using Kyber768 for quantum-resistant key exchange with LZ4 compression to offset PQC's bandwidth overhead. The compress-before-encrypt approach reduces payload size while maintaining NIST Level 3 security.

\textbf{Fourth}, we implemented and benchmarked this architecture, demonstrating practical viability with significant bandwidth savings for batched payloads while maintaining sub-millisecond processing times.

\section*{Answer to the Research Question}

The central question was: \textit{Can compression effectively offset PQC overhead for practical quantum-resistant IoT communications?}

Our findings provide an affirmative answer:
\begin{itemize}
    \item LZ4 compression reduces typical IoT sensor data by 50-75\%
    \item Combined with Kyber768, bandwidth savings exceed 60\% for batched payloads
    \item Processing overhead remains within acceptable bounds for microcontroller-class devices
    \item LZ4's low memory footprint (16 KB) coexists with Kyber's stack requirements
\end{itemize}

\section*{Limitations}

\begin{itemize}
    \item \textbf{Simulated PQC}: Due to liboqs library availability, some PQC operations were simulated
    \item \textbf{Desktop benchmarks}: Testing on general-purpose hardware rather than actual IoT microcontrollers
    \item \textbf{Single KEM focus}: Concentrated on Kyber; alternatives like NTRU may offer different trade-offs
\end{itemize}

\section*{Future Work}

\begin{itemize}
    \item \textbf{Hardware implementation}: Optimized implementations for ESP32, STM32, nRF52 microcontrollers
    \item \textbf{Energy analysis}: Comprehensive power consumption measurements on battery-powered devices
    \item \textbf{Protocol integration}: Integrating the approach into MQTT, CoAP, and LwM2M protocols
    \item \textbf{Post-quantum signatures}: Extending to include Dilithium for complete quantum-resistant security
\end{itemize}

\section*{Final Remarks}

The quantum threat to current cryptography is an approaching reality requiring proactive preparation. The ``harvest now, decrypt later'' strategy means data transmitted today using classical cryptography may be compromised once quantum computers arrive.

This thesis demonstrates that the transition to quantum-resistant security need not sacrifice efficiency. By combining LZ4 compression with Kyber768, we achieve practical quantum-resistant IoT security with acceptable overhead. LZ4's minimal memory footprint and extreme speed make it ideal for the resource-constrained endpoints that characterize IoT deployments.

\vspace{1cm}

\begin{flushright}
\textit{``The best time to prepare for quantum computing was yesterday.\\
The second best time is today.''}
\end{flushright}
