% ============================================================================
% CHAPTER 4: COMBINED APPROACH
% ============================================================================

\chapter{Combined Approach: PQC with Compression}
\label{chap:combined}

\section{Introduction}

This chapter presents our architecture combining Kyber768 (Chapter 2) with LZ4 compression (Chapter 3). The key insight: \textbf{compress data before encrypting} to offset PQC's bandwidth overhead while maintaining quantum resistance.

\begin{figure}[H]
\centering
\includegraphics[width=0.95\textwidth]{figures/system_architecture.png}
\caption{System architecture for combined compression and PQC approach}
\label{fig:system_arch}
\end{figure}

\section{Why Compress Before Encrypting}

Encryption produces output indistinguishable from random noise. Good ciphers eliminate all patterns---making compressed output incompressible. Therefore:

\begin{equation}
\text{Plaintext} \xrightarrow{\text{Compress}} \text{Compressed} \xrightarrow{\text{Encrypt}} \text{Ciphertext}
\end{equation}

Compressing \textit{after} encryption yields zero benefit. Security is unaffected: Kyber's security relies on Module-LWE hardness, not plaintext entropy.

\begin{figure}[H]
\centering
\includegraphics[width=0.8\textwidth]{figures/bandwidth_breakdown.png}
\caption{Bandwidth reduction through compression before encryption}
\label{fig:bandwidth_breakdown}
\end{figure}

\section{System Architecture}

\subsection{Overview}

\begin{figure}[H]
\centering
\begin{tabular}{|c|}
\hline
\\[0.2cm]
\textbf{IoT Device (Sender)} \\[0.3cm]
\framebox[0.8\textwidth]{
\begin{minipage}{0.75\textwidth}
\centering
Sensor Data → \textbf{Compress (LZ4)} → \textbf{Encrypt (Kyber768 + AES)} → Transmit
\end{minipage}
} \\[0.5cm]
$\downarrow$ \textit{Wireless Channel (LoRa, NB-IoT, etc.)} $\downarrow$ \\[0.5cm]
\textbf{Gateway/Server (Receiver)} \\[0.3cm]
\framebox[0.8\textwidth]{
\begin{minipage}{0.75\textwidth}
\centering
Receive → \textbf{Decrypt} → \textbf{Decompress (LZ4)} → Process
\end{minipage}
} \\[0.2cm]
\\
\hline
\end{tabular}
\caption{High-level system architecture}
\label{fig:system_architecture}
\end{figure}

\subsection{Sender Workflow}

\begin{algorithm}[H]
\caption{Sender: Compress and Encrypt}
\label{alg:send_workflow}
\begin{algorithmic}[1]
\Require Sensor data $D$, Receiver's public key $pk$
\Ensure Transmitted message $M$
\State $D_{compressed} \gets \Call{LZ4Compress}{D}$
\State $(ciphertext, sharedSecret) \gets \Call{KyberEncapsulate}{pk}$
\State $symmetricKey \gets \Call{HKDF-SHA256}{sharedSecret}$
\State $D_{encrypted} \gets \Call{AES-GCM-Encrypt}{symmetricKey, D_{compressed}}$
\State $M \gets ciphertext \| D_{encrypted}$
\State \Call{Transmit}{$M$}
\end{algorithmic}
\end{algorithm}

\subsection{Message Format}

\begin{table}[H]
\centering
\caption{Message format with Kyber768}
\label{tab:message_format}
\begin{tabular}{lcc}
\toprule
\textbf{Component} & \textbf{Size} & \textbf{Description} \\
\midrule
Kyber768 ciphertext & 1,088 B & Encapsulated shared secret \\
AES-GCM nonce & 12 B & Unique per message \\
Encrypted data & Variable & LZ4-compressed + encrypted payload \\
AES-GCM tag & 16 B & Authentication tag \\
\midrule
\textbf{Fixed Overhead} & \textbf{1,116 B} & Per message \\
\bottomrule
\end{tabular}
\end{table}

\section{Algorithm Selection Justification}

\subsection{Kyber768}

Selected for:
\begin{itemize}
    \item \textbf{NIST Standard}: FIPS 203 (ML-KEM)
    \item \textbf{Security Level 3}: Equivalent to AES-192, appropriate for 10-20 year protection
    \item \textbf{Balanced size}: 1,184 B public key, 1,088 B ciphertext
    \item \textbf{Fast operations}: Sub-millisecond on ARM Cortex-M4
\end{itemize}

\subsection{LZ4}

Selected for IoT endpoints (see Chapter 3):
\begin{itemize}
    \item \textbf{Minimal memory}: 16 KB (coexists with Kyber's stack)
    \item \textbf{Extreme speed}: 500+ MB/s compression, 2000+ MB/s decompression
    \item \textbf{Energy efficient}: Low CPU time minimizes battery drain
    \item \textbf{Good compression}: 40-60\% reduction on JSON sensor data
\end{itemize}

\section{Expected Performance}

\subsection{Bandwidth Model}

Let $D$ = original data size, $r$ = compression ratio, $O$ = 1,116 B fixed overhead.

\textbf{Bandwidth savings}:
\begin{equation}
\text{Savings} = \frac{(1-r) \cdot D}{D + O}
\end{equation}

\begin{table}[H]
\centering
\caption{Expected bandwidth savings with LZ4 (assuming 50\% compression)}
\label{tab:theoretical_savings}
\begin{tabular}{rrrrc}
\toprule
\textbf{Original} & \textbf{No Compress} & \textbf{With LZ4} & \textbf{Savings} \\
\midrule
500 B & 1,616 B & 1,366 B & 15.5\% \\
1,000 B & 2,116 B & 1,616 B & 23.6\% \\
2,000 B & 3,116 B & 2,116 B & 32.1\% \\
5,000 B & 6,116 B & 3,616 B & 40.9\% \\
10,000 B & 11,116 B & 6,116 B & 45.0\% \\
\bottomrule
\end{tabular}
\end{table}

\textbf{Key observation}: Larger payloads benefit more since fixed overhead is amortized.

\subsection{Comparison with Classical Cryptography}

\begin{table}[H]
\centering
\caption{PQC + LZ4 vs classical cryptography}
\label{tab:classical_vs_pqc}
\begin{tabular}{lccc}
\toprule
\textbf{Approach} & \textbf{Overhead} & \textbf{1 KB Payload} & \textbf{Security} \\
\midrule
ECDH + AES & 92 B & 1,092 B & Quantum-vulnerable \\
Kyber768 + AES (no compression) & 1,116 B & 2,116 B & Quantum-resistant \\
\rowcolor{green!10}
Kyber768 + AES + LZ4 & 1,116 B & 1,616 B & Quantum-resistant \\
\bottomrule
\end{tabular}
\end{table}

With LZ4, PQC is only ~48\% larger than classical cryptography while providing quantum resistance.

\section{Optimization Strategies}

\subsection{Session Keys}

For frequent communication, establish session key once:
\begin{enumerate}
    \item First message: Full Kyber exchange (1,088 B ciphertext)
    \item Subsequent messages: Use derived session key (0 B PQC overhead)
\end{enumerate}

\subsection{Batching}

Combine multiple sensor readings per transmission:

\begin{table}[H]
\centering
\caption{Batching efficiency}
\label{tab:batching}
\begin{tabular}{lccc}
\toprule
\textbf{Strategy} & \textbf{Messages/hour} & \textbf{Bytes/hour} \\
\midrule
Individual (100 B each) & 60 & 69,060 B \\
Batch of 10 (1 KB) & 6 & 9,696 B \\
Batch of 60 (6 KB) & 1 & 4,116 B \\
\bottomrule
\end{tabular}
\end{table}

\section{Security Analysis}

\begin{itemize}
    \item \textbf{Key Exchange}: Kyber768 provides IND-CCA2 security against quantum adversaries
    \item \textbf{Encryption}: AES-256-GCM provides authenticated encryption
    \item \textbf{Compression}: Does not weaken security (Kyber's security is independent of plaintext entropy)
\end{itemize}

\section{Chapter Conclusion}

Our combined architecture uses Kyber768 for quantum-resistant key exchange with LZ4 compression to minimize bandwidth. LZ4's low memory footprint (16 KB) and extreme speed make it ideal for IoT endpoints, while compression offsets PQC's size overhead. Chapter 5 presents implementation and benchmarks.
