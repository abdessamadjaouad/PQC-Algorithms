% ============================================================================
% ABSTRACT
% ============================================================================

\chapter*{Abstract}
\addcontentsline{toc}{chapter}{Abstract}

The Internet of Things (IoT) is growing rapidly, with billions of devices connected worldwide. However, these devices face two major challenges: they have limited resources (energy, memory, and bandwidth), and current security methods will become vulnerable when quantum computers become powerful enough.

This thesis addresses a key question: \textit{How can we protect IoT communications from quantum attacks while keeping bandwidth usage low?}

We propose a solution combining Post-Quantum Cryptography (PQC) with data compression. Our approach uses Kyber768, the NIST-approved post-quantum key encapsulation mechanism, together with LZ4 compression---chosen specifically for resource-constrained IoT devices due to its minimal memory footprint and energy efficiency.

The key insight is simple: compress data before encrypting it. Our benchmarks on realistic IoT sensor data demonstrate that this combined approach achieves significant bandwidth savings while maintaining sub-5ms processing times suitable for IoT devices. LZ4's extremely fast compression and decompression speeds make it ideal for battery-powered sensors where computational energy must be minimized.

\vspace{1cm}

\textbf{Keywords:} Internet of Things, Post-Quantum Cryptography, Compression, Kyber, LZ4, Bandwidth Optimization, NIST Standards

\newpage
