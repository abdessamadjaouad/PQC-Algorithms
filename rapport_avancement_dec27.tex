\documentclass[12pt,a4paper]{article}

% Packages
\usepackage[utf8]{inputenc}
\usepackage[T1]{fontenc}
\usepackage[english]{babel}
\usepackage{geometry}
\usepackage{graphicx}
\usepackage{hyperref}
\usepackage{booktabs}
\usepackage{enumitem}
\usepackage{fancyhdr}
\usepackage{xcolor}
\usepackage{array}

% Page geometry
\geometry{margin=2.5cm}

% Header/Footer
\pagestyle{fancy}
\fancyhf{}
\rhead{Progress Report - December 2025}
\lhead{Lightweight PQC for IoT}
\rfoot{Page \thepage}

% Colors
\definecolor{ensamblue}{RGB}{0,51,102}

% Hyperref setup
\hypersetup{
    colorlinks=true,
    linkcolor=ensamblue,
    filecolor=magenta,
    urlcolor=blue,
    citecolor=ensamblue
}

\begin{document}

% ============================================
% TITLE PAGE
% ============================================
\begin{titlepage}
    \centering
    
    % Logo ENSAM
    \includegraphics[width=6cm]{ensam_logo.png}
    
    \vspace{1.5cm}
    
    {\large \textbf{Master's Degree - 2nd Year}}\\[0.2cm]
    {\large Big Data and Internet of Things}\\[0.2cm]
    {\large Academic Year 2024-2025}
    
    \vspace{2cm}
    
    \rule{\textwidth}{1.5pt}\\[0.4cm]
    {\LARGE \textbf{Progress Report \#2}}\\[0.3cm]
    {\Large \textbf{Optimization of IoT Communications through Compression and Lightweight Post-Quantum Cryptography}}\\[0.4cm]
    \rule{\textwidth}{1.5pt}
    
    \vspace{8cm}
    
    \begin{minipage}{0.45\textwidth}
        \begin{flushleft}
            \textbf{Student:}\\
            Abdessamad JAOUAD\\
            M2 Big Data \& IoT
        \end{flushleft}
    \end{minipage}
    \hfill
    \begin{minipage}{0.45\textwidth}
        \begin{flushright}
            \textbf{Supervisor:}\\
            Prof. Ibrahim GUELZIM\\
            ENSAM Casablanca
        \end{flushright}
    \end{minipage}
    
    \vfill
    
    {\large December 2025}
    
\end{titlepage}

% ============================================
% INTRODUCTION
% ============================================
\section{Introduction}

This is the second progress report for my research project on \textbf{``Optimization of IoT Communications through Compression and Lightweight Post-Quantum Cryptography''}. This report covers the period from December 11 to December 27, 2025.

The project focuses on investigating how Post-Quantum Cryptography (PQC) algorithms have been adapted for resource-constrained IoT devices. The goal is to produce a concise, accessible research survey that explains PQC for IoT without excessive mathematical complexity.

% ============================================
% WORK COMPLETED
% ============================================
\section{Work Completed}

\subsection{Research on NIST-Standardized PQC Algorithms}

I conducted in-depth research on the four main NIST-standardized post-quantum algorithms:

\begin{itemize}[nosep]
    \item \textbf{ML-KEM (Kyber)}: Key Encapsulation Mechanism based on Module-LWE, standardized as FIPS 203
    \item \textbf{ML-DSA (Dilithium)}: Digital Signature Algorithm based on Module-LWE/SIS, standardized as FIPS 204
    \item \textbf{SLH-DSA (SPHINCS+)}: Hash-based signature scheme, standardized as FIPS 205
    \item \textbf{Falcon}: Compact NTRU-lattice-based signatures (upcoming standard)
\end{itemize}

I focused particularly on \textbf{lattice-based algorithms} (Kyber, Dilithium, Falcon) as they offer the best balance of security and performance for IoT applications.

\subsection{Key Insights Gained}

\subsubsection{Algorithm Characteristics}

I learned the key differences between the algorithms, including their key sizes, signature sizes, and performance trade-offs:

\begin{table}[h]
\centering
\small
\begin{tabular}{@{}lccc@{}}
\toprule
\textbf{Algorithm} & \textbf{Type} & \textbf{Public Key} & \textbf{Signature/Ciphertext} \\
\midrule
ML-KEM-768 & KEM & 1,184 B & 1,088 B \\
ML-DSA-65 & Signature & 1,952 B & 3,293 B \\
Falcon-512 & Signature & 897 B & 666 B \\
SLH-DSA-128s & Signature & 32 B & 7,856 B \\
\bottomrule
\end{tabular}
\caption{Comparison of PQC algorithm sizes (NIST Level 2/3)}
\end{table}

\subsubsection{Why Lattice-Based Algorithms are Preferred for IoT}

I understood why lattice-based schemes are the leading choice for IoT:
\begin{itemize}[nosep]
    \item Efficient polynomial operations via Number Theoretic Transform (NTT)
    \item Reasonable key and signature sizes (compared to code-based or hash-based)
    \item Good performance on ARM Cortex-M class microcontrollers
    \item Well-studied security foundations (LWE, SIS problems)
\end{itemize}

\subsection{Lightweight Optimization Techniques}

I researched how PQC algorithms are optimized for constrained IoT devices:

\begin{itemize}[nosep]
    \item \textbf{NTT optimization}: Fast polynomial multiplication reducing complexity from $O(n^2)$ to $O(n \log n)$
    \item \textbf{Memory management}: In-place operations, buffer reuse, streaming hash computation
    \item \textbf{Role-based design}: IoT devices only verify signatures; servers handle signing
    \item \textbf{Hardware acceleration}: Leveraging AES-NI and SHA accelerators when available
    \item \textbf{Parameter selection}: Using lower security levels (NIST Level 1) when appropriate
\end{itemize}

\subsection{IoT Constraints Analysis}

I studied the specific constraints of IoT devices and their implications for PQC:

\begin{table}[h]
\centering
\small
\begin{tabular}{@{}lcc@{}}
\toprule
\textbf{Device Class} & \textbf{RAM} & \textbf{PQC Feasibility} \\
\midrule
Class 0 (Sensor nodes) & $<$10 KB & Very limited \\
Class 1 (Smart meters) & $\sim$10 KB & Limited (verify only) \\
Class 2 (Wearables) & $\sim$50 KB & Feasible with optimization \\
Class 3 (Gateways) & $>$256 KB & Full PQC support \\
\bottomrule
\end{tabular}
\caption{IoT device classes and PQC compatibility}
\end{table}

\subsection{Document Writing Progress}

I made significant progress on writing my research survey:

\begin{itemize}[nosep]
    \item Created the \textbf{document outline and structure}
    \item Wrote \textbf{introduction and overview sections}
    \item Drafted sections for all four main algorithms (Kyber, Dilithium, SPHINCS+, Falcon)
    \item Wrote about \textbf{IoT constraints} and how PQC addresses them
\end{itemize}

The document is designed to be concise and accessible, focusing on practical insights rather than heavy mathematical details.

% ============================================
% CHALLENGES
% ============================================
\section{Challenges Encountered}

\begin{enumerate}[nosep]
    \item \textbf{Mathematical complexity}: Understanding the underlying mathematical concepts (lattices, LWE, SIS, polynomial rings) required significant effort
    \item \textbf{Algorithm derivation}: Connecting the abstract mathematical problems to the concrete algorithm implementations was challenging
    \item \textbf{Balancing depth vs. accessibility}: Finding the right level of detail for a practical survey without oversimplifying, or overwhelming the reader with the mathematical details
\end{enumerate}

% ============================================
% NEXT STEPS
% ============================================
\section{Next Steps}

The final report is due on the week of \textbf{January 5, 2026}. The remaining tasks are:

\begin{enumerate}[nosep]
    \item \textbf{Benchmark research}: Study real performance data from pqm4 project and academic papers
    \item \textbf{Library exploration}: Document available PQC libraries (liboqs, PQClean, pqm4, wolfSSL)
    \item \textbf{Comparison tables}: Create comprehensive comparison tables for the survey
    \item \textbf{Document completion}: Finalize all sections and polish the writing
    \item \textbf{Final review}: Proofread and prepare for submission
\end{enumerate}

% ============================================
% CONCLUSION
% ============================================
\section{Conclusion}

During this reporting period, I have made substantial progress on both the research and writing aspects of my project. I now have a solid understanding of the main NIST-standardized PQC algorithms, their optimization techniques for IoT, and the constraints of embedded devices.

The next phase will focus on incorporating real benchmark data and completing the final version of my research survey before the January 5th deadline.

\vspace{5cm}

\noindent\textbf{Submitted by:} Abdessamad JAOUAD\\
\textbf{Date:} December 27, 2025\\
\textbf{Supervisor:} Prof. Ibrahim GUELZIM

\end{document}
